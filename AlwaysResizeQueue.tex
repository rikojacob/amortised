\documentclass{tufte-handout}
\usepackage{tikz}\usetikzlibrary{decorations.pathreplacing,positioning,chains}
\usepackage{tipa}
\usepackage{color}
\usepackage{listings}
\usepackage{amsmath}
\usepackage{booktabs}

\input{vc.tex}


\usepackage[lastexercise,answerdelayed]{exercise}
\setlength{\Exesep}{.5ex}
\setlength{\Exetopsep}{1em}
\renewcommand{\ExerciseListName}{Exercise}
\renewcommand{\AnswerListName}{}
\renewcommand{\ExerciseHeaderTitle}{(\emph{\ExerciseTitle.})\ }
\renewcommand{\ExerciseListHeader}{\ExerciseHeaderDifficulty%
\textbf{\ExerciseListName\ExerciseHeaderNB.}\ \ExerciseHeaderTitle%
\ExerciseHeaderOrigin\ignorespaces}
\renewcommand{\AnswerListHeader}{\textbf{\ExerciseHeaderNB.\ }}

\title{Amortised analysis: Always Resizing Queue}
\author{Riko Jacob}
\date{\small Revision {\tt \GITAbrHash}$\ldots$, \GITAuthorDate, \GITAuthorName}
\begin{document}
\maketitle

\begin{abstract}
  This note analyzes the implementation AlwaysResizeQueue.java,
  a FIFO queue that avoids wrap around by always creating a new array of twice the current size when the end of the current array is reached.
\end{abstract}

\begin{figure}
\begin{tikzpicture}[remember picture,overlay] 
  \draw [fill=orange!30]
  (current page.south west) rectangle (current page.north east); 
\end{tikzpicture}

Algorithm A.1 

\small
\begin{lstlisting}[basicstyle=\ttfamily,backgroundcolor=\color{white},
  frame=single,rulecolor=\color{gray!20},framesep=10pt, linewidth=12cm]
public class AlwaysResizeQueue<Item> implements Iterable<Item> {
    private Item[] q;       // queue elements
    private int first;      // index of first element of queue
    private int last;       // index of next available slot


    public AlwaysResizeQueue() {
        q = (Item[]) new Object[4];
        first = 0;
        last = 0;
    }

    public boolean isEmpty() {
        return first == last;
    }

    public int size() {
        return last-first;
    }

    // resize the underlying array
    private void resize() {
        int n = last - first;
        Item[] copy = (Item[]) new Object[2*(n + 1)];
        for (int i = 0; i < n; i++) {
            copy[i] = q[(first + i) ];
        }
        q = copy;
        first = 0;
        last  = n;
    }

    public void enqueue(Item item) {
        // copy to new array if end reached
        if (last == q.length) resize();   // resize to double current size
        q[last++] = item;                 // add item
    }

    public Item dequeue() {
        if (isEmpty()) throw new NoSuchElementException("Queue underflow");
        Item item = q[first];
        q[first] = null;                            // to avoid loitering
        first++;
        // shrink size of array if necessary
        if ( last > first + 4 && last - first < q.length/4) resize(); 
        return item;
    }
}
\end{lstlisting}
\caption{\label{fig:impl}Implementation of Always Resize Queue}
\end{figure}


\begin{quote}
  {\bf Proposition 1.} In the Always Resize Queue implementation, all operations take amortized constant time, beginning from an initialized data structure.
\end{quote}

We use the localised piggy bank argument with the additional property that we can read of the number of available coins from the current state of the data structure.
This is also known as the potential method, where the potential of the data structure is the number of coins available in a certain state.

We use the variable $n = \texttt{last} - \texttt{first}$ for the current size of the data structure.

We observe that the data structure maintains the invariant $\texttt{last} \geq \texttt{q.length/2} -1$. 
\begin{quote}
  {\bf Potential.} The data structure has $\texttt{first} + 2(\texttt{last} - \texttt{q.length}/2 -1)$ coins in the piggy bank.
  One coin allows to copy 2 elements and create 4 slots in an array, i.e. stands for constant time.
\end{quote}

Observe that the actual cost of resize is $n$ copy operations and creating an array of $2n+2$ elements, i.e., $n$ coins.
After the resize operation, there are no coins in the piggy bank.

It remains to show that the all operations of the interface (not \texttt{resize}) take amortized constant time.
This is obvious as long as resize is not called, so it suffices to consider
\begin{description}
\item[\texttt{enqueue}]
  Without resize, there is constant work and two coins need to be paid.
  A resize happens when reaching the end of the array, hence $\texttt{last} - \texttt{q.length}/2-1 = \texttt{q.length}/2-1 \ge n/2 -1$.
  Hence, there are at least $n-2$ coins available to pay for the resize, and the resize has constant amortized cost.
\item[\texttt{dequeue}]
  Without resize, there is constant work and 1 coin to be paid.
  A resize happens if
  \[ \texttt{last} - \texttt{first} < \texttt{q.length/4} \]
  which, together with the invariant, implies $\texttt{first} \ge \texttt{q.length/4}$.
  Hence, the coins from \texttt{first} can pay at least for the elements in positions 
  $\texttt{q.length/4}\ldots \texttt{q.length/2}$, while the coins from \texttt{last} can pay for the positions   $\texttt{q.length/2}\ldots \texttt{last}$.

  \pagebreak
  We can also just do the calculus:
  \[ \texttt{q.length/4} > \texttt{last} - \texttt{first} \ge \texttt{q.length/2} -1 - \texttt{first} \]
  \[  \texttt{first} >  \texttt{q.length/4} -1 \]
  \[  \texttt{q.length/2} - \texttt{first}  <  \texttt{q.length/4} + 1 < \texttt{first} +2 \]
  \[ n = \texttt{last} - \texttt{first} = (\texttt{last} -\texttt{q.length/2} -1) + (\texttt{q.length/2} +1 - \texttt{first}) \le\]
  \[\le  2(\texttt{last} - \texttt{q.length}/2) + \texttt{first}  +3 \]
  Hence, the coins in the piggy bank can pay for most of the cost of the resize, but a constant 3 that is payed by the dequeue.
\end{description}

\end{document}